% !TeX spellcheck = en_US
%\documentclass[11pt,a4paper,UTF8]{report}
\documentclass[11pt,a4paper,UTF8]{ctexart}
\usepackage{setspace}
% \usepackage{titlesec}
\renewcommand{\baselinestretch}{1.5}
\usepackage[a4paper, left=3.17cm, right=3.17cm, top=2.54cm, bottom=2.54cm]{geometry}
\usepackage{graphicx}
\usepackage{float}
\usepackage{amsmath}



\title{模拟电子技术研究性学习报告}
\author{廖琛 17211401}
\date{\today}
\begin{document}
	\begin{figure}
	\centering
	\includegraphics[width=0.7\linewidth]{Pictures/BJTU_logo}
	\end{figure}

	
	\maketitle
	\newpage
	\tableofcontents
	\newpage
	\section{前言}
		\subsection{选用题目}
			\textbf{实用型多级放大电路研究:}
	
			了解多级放大电路的一般组成结构和级间耦合方式,分析不同耦合方式的特点。
	
			分析常见的二级放大电路(共射-共射、共集-共射、共射-共集、共射-共基)的特性,并指出各自适用的场合。选择一个实用型多级(三级及以上)放大电路,分析其组成及工作原理,利用仿真软件,模拟分析其输入输出特性。
		\subsection{注意事项}
			本报告中\textbf{默认所有三极管均相同},即其参数即特性均一致; 默认三极管$r'_{bb}$非常小,放大倍数$\beta$为数十到数百倍。

	\section{共射-共射放大电路}
		\subsection{分析}
			\begin{figure}[H]
				\centering
				\includegraphics[width=0.7\linewidth]{Pictures/EEnpn}
				\caption{共射-共射放大电路}
				\label{fig:eenpn}
			\end{figure}

			\subsubsection{静态点分析}
			对Q1进行分析:
			\begin{align}
				U_{B1} &= \frac{R_2}{R_1+R_2}V_{cc} \\	
				U_{E1} &= U_{B1} - U_{BE1} \\
				I_{C1} &\approx I_{E1} = \frac{U_{E1}}{R_4} \\
				U_{C1} &= V_{CC} - I_{C1}R_3 \nonumber\\ 
				&= V_{CC} - \frac{R_3}{R_4}U_{E1} \\
				U_{CE1} &= U_{C1} - U_{E1} \nonumber\\
				&= V_{CC} - (1+\frac{R_3}{R_4})[\frac{R_2}{R_1+R_2}V_{CC}-U_{BE1}] \nonumber\\
				&= [1-\frac{(R_3+R_4)R_2}{(R_1+R_2)R_4}]V_{CC} + \frac{R_3+R_4}{R_4}U_{BE}
			\end{align}
			
			对Q2进行分析:
				
			\begin{align}
				U_{B2} &\approx U_{C1} = V_{CC} - \frac{R_3}{R_4}U_{E1} \\
				U_{E2} &= U_{B2} - U_{BE2} \\
				I_{C2} &\approx I_{E2} = \frac{U_{E2}}{R_6} \\
				U_{C2} &= V_{CC} - I_{E2}R_5{}\nonumber \\
				&= V_{CC} - \frac{R_5}{R_6}U_{E2} \\
				U_{CE2} &= U_{C2} - U_{E2} \nonumber\\
				&= V_{CC} - (1+\frac{R_5}{R_6})(V_{CC} - \frac{R_3}{R_4}U_{E1}-U_{BE2}) \nonumber\\
				&= [\frac{R_2R_3}{R_4(R_1+R_2)}-\frac{R_5}{R_6}]V_{CC} + \frac{R_3+R_4}{R_4}U_{BE}
			\end{align}
			\subsubsection{交流分析}
			\begin{figure}[H]
				\centering
				\includegraphics[width=0.7\linewidth]{Pictures/EEnpnEqual}
				\caption{共射-共射放大电路等效图}
				\label{fig:eenpnequal}
			\end{figure}
			
			\textbf{输入电阻:}
			\begin{align}
				h_{ie1} &= r'_{bb} + \frac{26mV}{I_{B1}(mA)}h_{fe} \\
				R_i &= R_1 // R_2 // h_{ie1} \label{equ:EERi}
			\end{align}
			
			由式(\ref{equ:EERi})可知,共射-共射放大电路的输入电阻是较小的。
			
			\textbf{输出阻抗:}
			\begin{align}
				R_o = R_7 \label{equ:EERo}
			\end{align}
			
			由式(\ref{equ:EERo})可知,共射-共射放大电路输出电阻与负载有关,负载越大输出电阻越大。
			
			\textbf{增益:}
			\begin{align}
				A_U &= \frac{U_o}{U_i} \nonumber\\
				&= \frac{h_{fe}i_{ib2}R_7}{h_{ie1}i_{b1}} \nonumber\\
				&= \frac{h_{fe}\frac{h_{fe}i_{b1}R_3}{h_{ie2}}R_7}{h_{ie1}i_{b1}} \nonumber \\
				&= \frac{h^2_{fe}R_3R_7}{h_{ie1}h_{ie2}} \label{equ:EEAu}
			\end{align}
			
			由式(\ref{equ:EEAu})可知其$A_U$非常大。
			
			\subsection{适用场合}
				由以上分析可知:共射-共射放大电路输入电阻较小,输出电阻约等于负载电阻。因此我们可以得出,其适用于需要放大很多倍数,电源驱动能力强,且负载电阻较小时的场景。
			
	\section{共集-共射放大电路}
		\subsection{分析}
			\begin{figure}[H]
				\centering
				\includegraphics[width=0.7\linewidth]{Pictures/CEnpn}
				\caption{共集-共射放大电路}
				\label{fig:cenpn}
			\end{figure}

			\subsubsection{静态点分析}
				对Q1进行分析:
				\begin{align}
					U_{B1} &= \frac{R_2}{R_1+R_2}V_{CC} \\
					U_{E1} &= U_{B1} - U_{BE} \\
					I_{C1} &\approx I_{E1} = \frac{U_{E1}}{R_4} \\
					U_{C1} &= V_{CC} - I_{C1}R_3 \\
					U_{CE1} &= U_{C1} - U_{E1} \nonumber\\
					&= V_{CC} - \frac{R_3}{R_4}U_{E1} - U_{E1} \nonumber\\
					&= [1-\frac{R_2(R_3+R_4)}{R_4(R_1+R_2)}]V_{CC} + \frac{R_3+R_4}{R_4}V_{BE}
				\end{align}
				
				对Q2进行分析:
				\begin{align}
					U_{B2} &\approx U_{E1} = \frac{R_2}{R_1+R_2}V_{CC} - U_{BE} \\
					U_{E2} &= U_{B2} - U_{BE} \\
					I_{C2} &\approx I_{E2} = \frac{U_{E2}}{R_6} \\
					U_{C2} &= V_{CC} - I_{C2}R_5 \\
					U_{CE2} &= U_{C2} - U_{E2} \nonumber\\
					&= V_{CC} - \frac{R_5+R_6}{R_6}U_{E2} \nonumber\\
					&= [1-\frac{R_2(R_5+R_6)}{R_6(R_1+R_2)}]V_{CC} + \frac{2(R_5+R_6)}{R_6}U_{BE}
				\end{align}
			\subsubsection{交流分析}
				\begin{figure}[H]
					\centering
					\includegraphics[width=0.7\linewidth]{Pictures/CEnpnEqual}
					\caption{共集-共射放大电路等效图}
					\label{fig:cenpnequal}
				\end{figure}
			
			\textbf{输入电阻:}
			
			\begin{align}
				R_i &\approx R_1//R_2//[h_ie1+(R_4//h_ie2)] \label{equ:CERi}
			\end{align}
			
			由式(\ref{equ:CERi})可知,其输入电阻大。
			
			\textbf{输出电阻:}
			
			\begin{align}
				R_o &\approx R_5//R_7 \label{equ:CERo}
			\end{align}
			
			由式(\ref{equ:CERo})可知,其输出电阻较大。
			
			\textbf{增益:}
			
			\begin{align}
				A_U &= \frac{U_o}{U_i} \nonumber\\
				&= \frac{-h_{fe}i_{b2}(R_5//R_7)}{h_{fe}i_{b1}R_4} \nonumber\\
				&= -\frac{i_{b2}(R5//R7)}{i_{b1}R_4} \label{equ:CEAu}
			\end{align}
				
			由式(\ref{equ:CEAu})可知,其放大倍数一般。
		\subsection{适用场合}
			由以上分析可知,其输入电阻大,输出电阻较大,放大倍数较为可观。因此我们可以得出,其适用于电源驱动能力较差,负载电阻中等,且需要放大一定倍数的场景。

	\section{共射-共集放大电路}
		\subsection{分析}
			\begin{figure}[H]
				\centering
				\includegraphics[width=0.7\linewidth]{Pictures/ECnpn}
				\caption{共射-共集放大电路}
				\label{fig:ecnpn}
			\end{figure}
			\subsubsection{静态点分析}
			对Q1进行分析:
			\begin{align}
				U_{B1} &= \frac{R_2}{R_1+R_2}V_{cc} \\	
				U_{E1} &= U_{B1} - U_{BE1} \\
				I_{C1} &\approx I_{E1} = \frac{U_{E1}}{R_4} \\
				U_{C1} &= V_{CC} - I_{C1}R_3 \nonumber\\ 
				&= V_{CC} - \frac{R_3}{R_4}U_{E1} \\
				U_{CE1} &= U_{C1} - U_{E1} \nonumber\\
				&= V_{CC} - (1+\frac{R_3}{R_4})[\frac{R_2}{R_1+R_2}V_{CC}-U_{BE1}] \nonumber\\
				&= [1-\frac{(R_3+R_4)R_2}{(R_1+R_2)R_4}]V_{CC} + \frac{R_3+R_4}{R_4}U_{BE}
			\end{align}
			
			对Q2进行分析:
			\begin{align}
				U_{B2} &= U_{C1} = V_{CC} - \frac{R_3}{R_4}[\frac{R_2}{R_1+R_2}V_{CC}-U_{BE1}] \nonumber\\
				&= [1-\frac{R_2R_3}{(R_1+R_2)R_4}]V_{CC} + \frac{R_3}{R_4}U_{BE} \\
				U_{E2} &= U_{B2} - U_{BE} \\
				I_{C2} &\approx I_{E2} = \frac{U_{E2}}{R_5} \\
				U_{CE2} &= U_{C2} - U_{E2} \nonumber\\
				&= \frac{R_2R_3}{R_4(R_1+R_2)}V_{CC} + \frac{R_4-R_3}{R_4}U_{BE}
			\end{align}
				
			\subsubsection{交流分析}
			\begin{figure}[H]
				\centering
				\includegraphics[width=0.7\linewidth]{Pictures/ECnpnEqual}
				\caption{共射-共集放大电路等效图}
				\label{fig:ecnpnequal}
			\end{figure}
			
			\textbf{输入电阻:}
			\begin{align}
				R_i = R_1//R_2//h_{ie1} \label{equ:ECRi}
			\end{align}
			
			由式(\ref{equ:ECRi})可知,其输入电阻较小。
			
			\textbf{输出电阻:}
			\begin{align}
				R_o = R_5//R_6//(h_{ie2} + R_3) \label{equ:ECRo}
			\end{align}
			
			由式(\ref{equ:ECRo})可知,其输出电阻较大。
			
			\textbf{增益:}
			\begin{align}
				A_U &= \frac{U_o}{U_i} = \frac{h_{fe}i_{b1}R_3}{-h_{ie1}i_{b1}} \nonumber\\
				&= -\frac{h_{fe}R_3}{h_{ie1}} \label{equ:ECAu}
			\end{align}

			由式(\ref{equ:ECAu})可知,其放大倍数较大。

		\subsection{适用场合}
			由以上分析可知,其输入电阻较小,输出电阻较大,放大倍数较大。因此我们可以得出,其适用于电源驱动能力较强,负载电阻很小,且需要放大一定倍数的场景。

	\section{共射-共基放大电路}
		\subsection{分析}
			\begin{figure}[H]
				\centering
				\includegraphics[width=0.7\linewidth]{Pictures/EBnpn}
				\caption{共射-共基放大电路}
				\label{fig:ebnpn}
			\end{figure}
			\subsubsection{静态点分析}
			对Q1进行分析:
			\begin{align}
				U_{B1} &= \frac{R_2}{R_1+R_2}V_{CC} \\
				U_{E1} &= U_{B1} - U_{BE} \\
				I_{C1} &\approx I_{E1} = \frac{U_{E1}}{R_4} \\
				U_{C1} &= \frac{R_6}{R_3+R_6}V_{CC} - U_{BE} \\
				U_{CE1} &= U_{C1} - U_{E1} \nonumber\\
				&= (\frac{R_6}{R_3+R_6}-\frac{R_2}{R_1+R_2})V_{CC}
			\end{align}
			
			对Q2进行分析:
			\begin{align}
				U_{B2} &= \frac{R_6}{R_3+R_6}V_{CC} \\
				U_{E2} &= U+{B2} - U_{BE} \\
				I_{C2} &\approx I_{E2} = I_{C1} \\
				U_{C2} &= V_{CC} - I_{C2}R_5 \\
				U_{CE2} &= U_{C2} - U_{E2} \nonumber\\
				&= V_{CC} - \frac{R_5}{R_4}U_{E1} - \frac{R_6}{R_5+R_6}V_{CC} + U_{BE} \nonumber\\
				&= [1-\frac{R_2R_5}{R_4(R_1+R_2)} - \frac{R_6}{R_3+R_6}]V_{CC} + \frac{R_4+R_5}{R_4}U_{BE}
			\end{align}
			
			\subsubsection{交流分析}
				\begin{figure}[H]
					\centering
					\includegraphics[width=0.7\linewidth]{Pictures/EBnpnEqual}
					\caption{共射-共基放大电路等效图}
					\label{fig:ebnpnequal}
				\end{figure}
			
			\textbf{输入电阻:}
			
			\begin{align}
				R_i = R_1//R_2//h_{ie1} \label{equ:EBRi}
			\end{align}
			
			由式(\ref{equ:EBRi})可知其输入电阻较大
			
			\textbf{输出电阻:}
			
			\begin{align}
				R_o = R_5 \label{equ:EBRo}
			\end{align}
			
			由式(\ref{equ:EBRo})可知其输出电阻较大
			
			\textbf{增益:}
			
			\begin{align}
				A_U &= \frac{U_o}{U_i} \nonumber\\
				&= \frac{-h_{fe}i_{b2}(R_5//R_7)}{h_{ie1}i_{b1}} \label{equ:EBAu}
			\end{align}
			
			由式(\ref{equ:EBAu})可知其放大倍数较为可观
				
		\subsection{适用场合}
			由以上分析可知,其输入电阻较大,输出电阻大小,放大倍数较大。因此我们可以得出,其适用于驱动能力弱的电源放大高频信号。
			
		\section{电路设计}
		\subsection{概述}
		\begin{figure}[H]
			\centering
			\includegraphics[width=0.7\linewidth]{Pictures/MA}
			\caption{三级放大电路}
			\label{fig:mafreq}
		\end{figure}
		
		本电路通过三个共射放大电路级联的方式,可以将非常小的信号放大,最大放大倍数约为2.7k倍。
		
		\textbf{备注:}
		
		本实验设$ U_{BE} $=0.7V,基极内部等效电路$ r'_{bb} $ = 100$ \Omega $。
		
		\subsection{分析}
			\subsubsection{静态点分析}
			对Q1:
			\begin{align}
				U_{B1} &= 0.7V \nonumber\\
				U_{C1} &= \frac{R_2}{R_1+R_2+R_6}(V_{CC}-U_{B1}) \nonumber\\
				&= 4.24V \nonumber\\
				I_{B1} &= \frac{U_{C1}-U_{B1}}{R_2} \nonumber\\
				&= 6.32\mu{}A \nonumber\\
				I_{E1} &\approx I_{C1} = \frac{V_{CC}-U_{C1}}{R_1+R_6}-I_{B1} \nonumber\\
				&= 91\mu{}A \nonumber\\
				U_{CE1} &= U_{C1} - U_{E1} \nonumber\\
				&= 4.24V \nonumber
			\end{align}
			
			对Q2:
			\begin{align}
				U_{B2} &= 0.7V \nonumber\\
				U_{C2} &= \frac{R_3}{R_3+R_4}(V_{CC} - V_{B2}) \nonumber\\
				&= 4.248V \nonumber\\
				I_{B2} &= \frac{U_{C2} - U_{B2}}{R_3} \nonumber\\
				&= 13.14\mu{}A \nonumber\\
				I_{E2} &\approx I_{C2} \approx \frac{V_{CC}-U_{C2}}{R_4} \nonumber\\
				&= 227.8\mu{}A \nonumber\\
				U_{CE2} &= U_{C2} - U_{E2} \nonumber\\
				&= 4.248V \nonumber
			\end{align}
			
			对Q3:
			\begin{align}
				U_{B3} &= 0.7V \nonumber\\
				U_{C3} &= \frac{R_5}{R_5+R_L}(V_{CC} - U_B) \nonumber\\
				&= 4.28V \nonumber\\
				I_{B3} &= \frac{U_{C3} - U_{B2}}{R_5} \nonumber\\
				&= 0.239mA \nonumber\\
				I_{E3} &\approx I_{C3} \approx \frac{V_{CC}- U_{C3}}{R_L} \nonumber\\
				&= 11.25mA \nonumber\\
				U_{CE3} &= U_{C3} - U_{E3} \nonumber\\
				&= 4.28V \nonumber
			\end{align}
			\subsubsection{交流分析}
			\begin{figure}[H]
				\centering
				\includegraphics[width=0.7\linewidth]{Pictures/MAEqual}
				\caption{三级放大电路等效图}
				\label{fig:maequal}
			\end{figure}
			
			\begin{align}
				h_{ie1} &= r'_{bb} + \frac{26mA}{I_{B1}mA} \nonumber\\
				&\approx 4.11k\Omega{} \nonumber\\
				h_{ie2} &= r'_{bb} + \frac{26mA}{I_{B2}mA} \nonumber\\
				&\approx 1.98k\Omega{} \nonumber\\
				h_{ie3} &= r'_{bb} + \frac{26mA}{I_{B3}mA} \nonumber\\
				&\approx 0.1k\Omega{} \nonumber\\
				R_i &= 3.29k\Omega \\
				R_o &= 15k\Omega
			\end{align}
		\subsection{仿真结果}
		\begin{figure}[H]
			\centering
			\includegraphics[width=0.7\linewidth]{Pictures/MAfreq}
			\caption{交流分析}
			\label{fig:mafreq}
		\end{figure}
		
		从图(\ref{fig:mafreq})可知,其$ f_L \approx 2.51kHz $,$ f_H \approx 908.3kHz $, 带通约$ 900kHz $,最大放大倍数约为3700倍。由此可见,其放大倍数以及高频响应极为可观,可用于放大极小的信号。
	\newpage
\end{document}          
